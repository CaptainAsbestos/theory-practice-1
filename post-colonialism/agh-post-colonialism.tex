\documentclass[letterpaper,12pt,twoside]{article}
\usepackage[utf8]{inputenc}
\usepackage[T1]{fontenc}
\usepackage{mathptmx}
\usepackage{helvet}
%\renewcommand{\familydefault}{\sfdefault}
\usepackage{multicol}
\setlength{\columnsep}{0.33in}
\usepackage[margin=1in]{geometry}
\addtolength{\topmargin}{0.125in}
\usepackage{microtype}
\usepackage{hyperref}
\hypersetup{
    colorlinks=true,
    linkcolor=blue,
    filecolor=magenta,
    urlcolor=cyan,
}
\renewcommand\refname{Bibliography} %changes the References Section to whatever you like
\usepackage{csquotes}
%\usepackage[notes,backend=biber]{biblatex-chicago}

\usepackage{titling}
\usepackage{sectsty}
%\sectionfont{\fontfamily{phv}\selectfont}

\title{\textbf{Post-Colonial Theory}}
\author{Alex Horne \\ PS 6100 Theory \& Practice I}
\date{}

\usepackage{fancyhdr}
\pagestyle{fancy}
\chead{Post-Colonial Theory}
\rhead{Alex Horne}
\lhead{Theory \& Practice I}
%\setlength{\headheight}{40pt}
\cfoot{}
\fancyhf[FLE,FRO]{\thepage}


\begin{document}
\maketitle

\begin {multicols}{2}
\section*{Finnemore, 2003}

Finnemore seeks to answer the question of why states intervene in humanitarian crises caused by humans and which only affect foreign nationals. She states that the motivations of intervention cannot be analysed without awareness of norms relating to human rights, namely: who counts as "human," what counts as "legitimate" intervention, and what do states seek to accomplish by intervening (what is "success")? In her survey of history, she explores how these norms varied over time, beginning in the age of imperialism and culminating in the Balkan Crisis of the '90s.

She begins by detailing why she finds other IR theories inadequate for explaining why states intervene for humanitarian purposes. She specifically points to Somalia, Cambodia, and Bosnia/Kosovo as interventions which happened despite realist predictions. And before we jump to conclusions that these were liberal crusades, the intereveners had little interest in nation-building. Finnemore takes humanitarians at their word, but wants to uncover what "humanitarian" action means, why it happens, and how it interacts with other variables which influence foreign policy. Crucially, the norms around "humanitarianism" are related to other political concepts, such as sovereignty, legitimacy, and justice, which have their own set of constituitive norms.

Next, she traces the history of interventions to demonstrate that their nature has changed, beginning with the Greek War for Indpendence and ending with the Armenian Genocide. (Arguably, the Crusades might count as a "humanitarian intervention" in the least enlightened sense.) Although in most cases imperial geopolitics were what interested state governments, atrocities always catalysed intervention. Interestingly, even back then, interventions were often multi-lateral so as to deter any one empire from championing the cause of the oppressed to further its own agenda. Crucially, Finnemore identifies the Armenian Genocide as a case of non-intervention – precisely because the Prussian/German empire was courting the Ottomans, and the Russians felt no obligation to the Armenian Church. Thus, the genocide was an illustrative example of "non-intervention."

Finnemore moves on to unpacking the expanding scope of "humanity" in history. By this, she refers to how the community of moral concern has expanded from strict tribalism, to nationalism, to universal human rights which states are obliged to recognise and defend. While she illustrates that humanitarian civil society is not new within Europe, interventions were launched in the name of protecting non-Christians as time progressed. The abolition of slavery and decolonisation rewrote international norms regarding human rights, civilisation, and sovereignty, which in turn influenced how and why humanitarian interventions happened. By the time the UN formed, decolonisation was already underway – but post-colonial norms were institutionalised, which accelerated the process. \footnote{
  Finnemore doesn't give due credit to the role of critical ideology here. It was not merely that independence leaders in colonial territories were educated in the imperial core (such as Ghandi), but also that many adopted anti-imperialist ideologies such as Leninism in their struggle (such as Ho Chi Minh). While all benefited from evolving norms, the hard struggle for independence was waged differently in different countries. %Crucially, norms about human rights were used to justify intervention \textit{against} newly independent colonies such as Cuba, Vietnam, and Burkina Faso, in many cases unilatearlly. But by the strict definition of intervention she is using, these don't qualify – which I see as a shortcoming of her analysis.
}

Despite greater concern and awareness of human rights since the end of the Second World War, most interstate interventions were perceived not as humanitarian adventures but unilateral geopolitical acts by great powers. After the Soviet collapse, however, the US had free-reign to intervene anywhere (which suggests, to me, that Finnemore is overstating the importance of multi-lateralism as a catalyst for intervention), and at least attempted to cobble together international coalitions for aid missions. The decades since 2001, however, weaken any confidence I have in her hypothesis, but she would likely counter that the US expedition in the Persian Gulf is not considered legitimate – but what difference does that make? This passage in particular (presumably drafted in the years leading up to 2003) seems laughably out-of-touch in retrospect:
  \begin{displayquote}
    ...multilateralism in the present day has become insti­tutionalized in ways that make unilateral intervention, particularly inter­vention not justified as self-defense, unacceptably costly, not in material terms but in social and political terms.\footnote{\
      Finnemore, "The Purpose of Intervention", 75.
    }
  \end{displayquote}
How can one observe the flagrant and ongoing violation of human rights norms in Southwest Asia and conclude anything but that her hypothesis no longer applies to the US? While the second invasion of Iraq began as a nuclear-non-proliferation mission, it mutated into humanitarian nation-building after WMDs were not found.

But I digress. Finnemore recounts several cases throughout the Cold War in which minor powers intervened abroad, but rarely did they successfully invoke humanitarian interest as justification, even if that was an inevitably by-product. This was partly a result of the bipolar geostrategic landscape at the time: multilateralism across the Iron Curtain was nearly nonexistent.

Returning to the '90s, Finnemore observes that inteventions in this decade were all multilateral coalitions.

Finnemore also returns to cases of \textit{non-intervention}, seeking to explain why genocides and mass murders happened despite international law proscribing them. The Rwandan Genocide is the most recent example she considers. In this case, Western Powers rhetorically minimised the atrocities to avoid their treaty obligations to end the slaughter. These efforts revealed that states recognised humanitarian norms even though they sought to deny the implications of these norms. The resulting public outrage forced their hand for the rest of the decade, leading to US missions in Somalia, Haiti, and with NATO in the Balkans.\footnote{
  For the record, the US "humanitarian" intervention in Haiti ended with Clinton leaving Haitian refugees (fleeing US-trained death squads) to die of Malaria, exposure, and AIDS at "detention facilities" in Guantánamo Bay rather than allow them onto US soil and be required by law to entertain their asylum claims. Does this even count as lip service to humanitarianism? For more, see \href{https://doi.org/10.1215/01636545-1724751}{Paik, "Carceral Quarantine at Guantánamo"}.
}

A major methodological shortcoming of Finnemore's work is that her discussion of norms and interventions are unfortunately Eurocentric. While the subject of intervention eventually expands to non-white non-christians, the question of human suffering is only addressed when it becomes visible by Western audiences. Throughout her research, helpless victims exist in a semantic zone of non-being, to borrow Fanon's terminology. At no point does she contemplate how targeted groups regard humanitarian norms, intervention, and non-intervention. While these are clearly at a lower and richer level of analysis than state-to-state relations, they are \textit{vitally important} at explaining the success or failure of these expeditions. %As stated above, I find her confidence in multilateralism as a relevant variable in these interventions misplaced, and these perceptions on the ground-level might explain why.

Despite expecations I had from the introduction of this chapter, Finnemore does not elaborate much on the changing "success" conditions of humanitarian missions since the 19th century. She seemed to indicate that early one, European powers were content to install a new regime which answered either to an outside power or had autonomy within an abusive empire. The example of slavery abolition complicates this broad generalisation, since emancipation was an intra-state process for most places while the British treated it as a violation of maritime law to \textit{traffick} humans into bondage. The contemporary situation was at first to ameliorate humanitarian crises without nation building – but since 2003, the US has actively sought to rollback and replace human rights violators (where convenient for the US). During the Somalian Famine, American military leaders and diplomatic leaders disagreed about their fundamental purpose in the Horn, which she never touches on when it comes to changing norms of "success."

This piece is but one of many chapters in a larger book, so I would hope that Finnemore addresses these oversights in her other writing then or since then. With respect to this excerpt, however, I find her hypothesis theoretically wanting and historically antiquated.

\vfill\null
\columnbreak

\section*{Krasner, 2004}
Krasner diputes the notion that dysfunctional or failed states can save themselves or their societies without outside help; by the same token, he asserts that occupying powers must consider trusteeships and shared sovereignty as options for maintaining order in occupied territories.

The author splits the concept of sovereignty in three: international legal sovereignty, Westphalian, and domestic. This is to say that only sovereigns can recognise sovereigns, \textit{cuius regio, cuius religio}, and that sovereigns must guarantee the welfare of their citizens or subjects. He explains that this ideal is unrealistic in failed states, because leaders face perverse incentives in deteriorating political situations.

Krasner offers "shared sovereignty" as a remedy, using Soviet occupied Europe as an example of this concept's success in practice (he does not mention how the USSR was widely resented in Eastern Europe). He explains its appeal thus: elites within the minor state can use raw resources as a bargaining chip with major states and enjoy greater security as a junior partner of the trusteeship.

Krasner avoids discussing the effect of US sanctions on failed states, such as Interbellum Iraq. Economic sanctions worsen an already bad situation while hardening anti-Western resolve. Importantly for outside invaders (the only people Krasner wants to have agency), sanctions weaken a state to the point that it can be knocked over easily and placed under trusteeship. Sanctions and invasion are two sides of the same coin: two stages of a multi-stage process which vacillates with the partisanship of the US president. During Krasner's tenure as Director of Policy-Planning in the Bush White House (from 2005-2007), we can see how well his theory for shared-sovereignty worked in practice: not well at all. The De-Ba'athified state was resented by key ethnic constituencies in Iraq and correctly perceived as corrupt from top-to-bottom (this wasn't so much Krasner's fault as it was Achmed Chalabi's and the INC's). Because the US never bothered to legitimate 'shared sovereignty' with the provisional government through multilateral UN discussions, it undermined its own mission at every turn – something which Krasner found perfectly acceptable:
  \begin{displayquote}
    Under such an arrangement, the Westphalian/Vatellian sovereignty of the target polity would be violated, executive authority would be vested primarily with external actors, and international legal sovereignty would be suspended. \textit{There will not, however, be any effort to formalize through an international convention or treaty a general set of principles for such an option.} [emphasis added]\footnote{Krasner, "Shared Sovereignty", 89.}
  \end{displayquote}
While we all should be sceptical that multilateral trusteeship would be better than unilateral vassalage, Krasner \textit{never} considers how the people under occupation will perceive such an arrangement. And in 2004, Krasner must have been aware of how the 1919 Versailles Peace was perceived as a "stab in the back" by monarchists in Germany; this myth eventually made many others in the country miss a monarchy they were previously ambivalent about. No surprise, then, that people came to remember the Hussein regime more fondly \textit{after} the US took over the country in the name of trusteeship and democracy.

Perhaps a more apt comparison might be Qing-dynasty China throughout the Century of Humiliation. Not only was the country in the vice-grip of an opioid pandemic, but China was losing its central position in the global silver supply as the English cartels sold the drug for bullion, effectively looting the imperial treasury. Deflation and stagnation ensued. Losing control of their own monetary policy, Krasner would have argued that the Qing were unfit to govern their country, and the race to slice China up into spheres of influence is perfectly compatible with what he proposes for failed states in the 21st century. And importantly, it would behoove us to remember that the motivation for the Opium War was a commodity – tea – upon which Europe was dependent. Petrol – which is found underneath Kuwait and Iraq –  motivates US foreign policy to an equal extent.\footnote{For more on the role of silver in the fall of the Qing Dynasty, I recommend Xu, \textit{Empire of Silver}, 2021.} In the Chinese case, national humiliation consolidated national identity where none previously existed – a positive outcome, in my opinion. In Iraq, humiliating defeat and imperial subjugation have obliterated any chance of constructing a national identity which transcends religious tribalism.

This article amounts to nothing but a resumé written by a crackpot realist whose role within the American Empire is to give extraction-colonialism a veneer of intellectual rigour and humanitarian interest. Case examples he uses seem hilariously out-of-date even in 2004 while he ignores seminal conflicts in human history which would debunk his theoretical framework. Krasner has little worth saying or worth hearing;  even at the time of his writing, the world knew better than to credulously heed his advice.

\section*{Biswas}

I have very little to add to Biswas, other than to comment on the absence of Fanon in her survey of Post-Colonial IR Theory. I wish I had read this part \textit{before} reading Finnemore \& Krasner, because one can draw an almost one-to-one correspondence between what she writes about and the uninterrogated assumptions underlying both of the previous authors' work. Krasner is the cold cartesian whose dispassionate musings provide cover for neo-colonial violence; Finnemore is the well-meaning liberal for whom the Third World exists only as a captive to be saved.

The case study for Iran is also a wonderful theatre for applying post-colonial theory. While I'm glad Biswas gives proper due to the Iranian perspective, when it comes to nuclear weapons, it's easy to understand why a one-percent chance of nuclear war is treated as if its a ninety-nine percent chance (I say this fully recognising my white male privilege in the heart of the empire). This doesn't change the fact that the onus is on the US and its partners to denuclearise as well as the Islamic Republic – which seems to be the synthesis of Iran's legitimate security worries and the Western desire for non-proliferation.

One part which struck me was how much the US projects its own psycho-national insecurity (insofar as such a thing exists) upon Iran. Scholars write at length that nuclear weapons are symbols of might and national pride, that they shore up rulers with legitimacy problems, and that nuclear-seeking states will become strident and undeterrable. To me, this describes the United States \textit{perfectly}: nothing can deter its imperial expansionism, the ruling class is in an ongoing legitimacy crisis (owing to the legacy of slavery and the contradictions of capitalism), and US patriotism has become adulating military fetishism. And last, but not least, the US is the only nation in history to ever use nuclear weapons against an enemy – an enemy already on its knees with no way to retaliate in kind – and it dropped The Bomb to show rest the world that it was \textit{crazy enough to do so}. And it did so not once, \textit{but twice.} The US in practice is everything it fears about Iran: like all colonial masters, the one thing they fear most is their slaves "returning the favour." When the script is flipped, the US is what needs deterring, and Iran has been pushed to pursue its nuclear weapons out of desperation, even under the 2013 agreement (which was torn up by Trump).

This all emerges from a White Protestant entitlement which sprung from the original colonial encounter. The world was the white man's to inherit, for him to go forth and multiply, for him to clear the jungle and tame the earth. To the white lords, the resources of the world exist for \textit{their} benefit. To the colonizer, using those resources to the benefit of the people who live in those environs is tantamount to theft of rightful inheritance. This unstated axiom undergirds the West's continued prosperity; to reckon with it would be the unravelling of everything.

\end{multicols}

\pagebreak

\begin{thebibliography}{99\kern\bibindent}
  \makeatletter
    \let\old@biblabel\@biblabel
    \def\@biblabel#1{
      \old@biblabel{#1}\kern\bibindent
    }
    \let\old@bibitem\bibitem
    \def\bibitem#1{
      \old@bibitem{#1}\leavevmode\kern-\bibindent
    }
    \renewcommand\@biblabel[1]{}
    \makeatother
%% The previous lines are what give you the nice under-indent for turabian style and removes the Numbers

  \raggedright
  \bibliographystyle{chicago}
  %\bibliography{}

    \bibitem{finnemore} Finnemore, Martha. "The Purpose of Intervention: Changing Beliefs About the Use of Force." Ithaca; London: Cornell University Press, 2003. Accessed February 28, 2021. \texttt{http://www.jstor.org/stable/10.7591/j.ctt24hg32.}

    \bibitem{krasner} Krasner, Stephen D. "Sharing Sovereignty: New Institutions for Collapsed and Failing States." International Security 29, no. 2 (2004): 85-120. Accessed March 7, 2021. \texttt{http://www.jstor.org/stable/4137587.}

    \bibitem{paik} Paik, Naomi A. "Carceral Quarantine at Guantánamo: Legacies of US Imprisonment of Haitian Refugees, 1991-1994." \textit{Radical History Review} 115, Winter 2013: 142-168. Accessed March 6, 2021. \textsc{doi:} \texttt{10.1215/01636545-1724751.}

    \bibitem{xu} Xu, Jin. "The Late Qing: Collapsing in Chaos." In \textit{Empire of Silver: A New Monetary History of China}, 159-243. Translated by Stacey Mosher. New Haven: Yale University Press, 2021. Accessed March 7, 2021. \textsc{doi:} \texttt{10.2307/j.ctv1dv0vpw.7.}

\end{thebibliography}



\end{document}

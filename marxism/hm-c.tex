\documentclass[letterpaper,12pt]{article}
\usepackage[utf8]{inputenc}
\usepackage[T1]{fontenc}
\usepackage{mathptmx}
\usepackage{helvet}
%\renewcommand{\familydefault}{\sfdefault}
\usepackage{multicol}
\setlength{\columnsep}{0.5in}
\usepackage[margin=0.75in]{geometry}
\usepackage{microtype}
\usepackage{hyperref}

\usepackage{sectsty}
\sectionfont{\fontfamily{phv}\selectfont}

\title{\fontfamily{phv}\selectfont \textbf{Historical Materialism \\ \& Constructivism}}
\author{\fontfamily{phv}\selectfont Alex Horne \\ \fontfamily{phv} \selectfont PS 6100 Theory \& Practice I}
\date{\fontfamily{phv}\selectfont}

\usepackage{fancyhdr}
%\usepackage{fancyfoot}
\pagestyle{fancy}
\chead{\fontfamily{phv}\selectfont Historical Materialism \\ \& Constructivism}
\rhead{\fontfamily{phv}\selectfont Alex Horne\\}
\lhead{\fontfamily{phv}\selectfont Theory \& \\ Practice I}
%\cfoot{\fontfamily{phv}\selectfont \thepage}
%\setlength{\headheight}{40pt}
\cfoot{}
\fancyfoot[LE,RO]{\thepage}


\begin{document}
\maketitle

\begin {multicols}{2}
\section*{Lenin (1917)}

Vladimir Lenin is noteworthy as one of the few political theorists better known for praxis than theory. \textit{Imperialism} was written the April before the Bolshevik revolution overthrew the provisional Russian government and proclaimed the Soviet Union. In this book, Lenin argues against Marxist orthodoxy of the time, explaining the relationship between capital, state, and conquest as it appeared in his lifetime.

Empires come into existence in order to overcome the contradictions of mature industrial capitalism. The profit motive encouraged the capitalists to hoard wealth and compensate workers as little as possible while also requiring greater and greater production. Eventually, the circulation of goods and cash would grind to a halt, unless an outlet could be found for manufactured goods: colonies. Thus, the proletariat in the homeland could be bought off by the bourgeoisie using profits made at the frontier – the cost of this moderately improved standard of living was off-loaded onto someone out of sight and out of mind.

However, resolving class conflict in the home country did no necessarily resolve all political conflict. The Russian Empire, already a waning military power since the 1905 debacle against Japan, had already expended great amounts of blood and treasure in the war against the Central Powers. In Lenin's estimation, the Great War amounted to nought but a feud within the capitalist classes of the European Empires.  Lenin contended that it was a war of imperialist expansion, with the up-and-coming German Empire seeking colonial markets and resource access. The Entente powers refused to budge, preferring to use their armies to ensure their nations' control over the world. Communist opposition to the war, therefore, was resistance to imperialism as state policy, which required a militant vanguard party to raise class consciousness among the European Lumpens.

Lenin can and has been critiqued for presuming a permanent harmony of interests between the capitalist class and the imperial state; however, at the time, it would have been reasonable to conclude as much. The century of history since the First World War has demonstrated not that Lenin's analysis was false, but that the structure of the state would continue to mutate as technology advanced and society evolved. A more accurate declaration would have been, perhaps, that imperialism is merely the ``Highest Stage of Capitalism – so far.'' The basic model which Lenin proposed has not changed; modern empires now have much more sophisticated techniques by which they dominate territories for resource extraction. The US Empire supplanting Britain can be explained by the Americans preferring a less hands-on approach to local governance while retaining all the extractive mechanisms of the world economic system.

%\vfill\null
%\bigskip

\section*{Dos Santos (1970)}

Dos Santos describes the condition of economic dependence in which otherwise independent former colonies found themselves in the '70s. Underdevelopment of the Third World was not an obstacle to be overcome by modernisation; rather, underdevelopment was a necessary outcome of liberal international capitalism. The newest variety of dependence is technological: LDCs lack the know-how to sustain their own economies, but MNCs from DCs are willing to provide it... for a steep price. Countries which fail to integrate into the global capitalist system are branded backwards and unwilling to comply, never mind that their backwardness emerged as a result of global capitalism itself.

\section*{Wallerstein (1974)}

Wallerstein acknowledges that Marxism, no longer a permanently oppositional ideology, has entered international politics as a viable competitor to capitalism. Thus, the status quo has adjusted to accomodate its arrival, to position even anti-capitalist ideology within the world economic system. He rejects the proposition that states could "leapfrog" over capitalism into socialism, and he also rejects the notion that all states must pass through the capitalism to lay the groundwork for socialism.

For Wallerstein,  The only unit of analysis worth studying is the world system itself: states vie for status and wealth within that system, but it is fundamentally capitalistic. The division of labour between worker and manager is found everywhere because the market has become globalised; the only variation in modes of production is cultural. Borders on a map are just shields which capitalists use to compete with one another. (Hopf would vehemently dispute this point.)

Wallerstein identifies a tendency for a single, aggressive state to exploit the weaknesses within a world system and bring it under hegemonic control; if successful, it becomes a \textit{world empire}, like the Roman Mediterranean. The historical circumstances leading to the development of capitalism in Northwest Europe meant that industrial capitalism developed alongside wage-labour, but serfdom and chattel slavery were still necessary in peripheral territories to sustain the growth of the core. Wallerstein also coins the term \textit{semi-periphery} to describe nations which witnessed a political divide between the aristocracy and the bourgeoisie (he uses Monarchist Poland as his example). Whereas the British and Dutch burgers worked hand in glove with imperial enterprises, the Polish King never had such interests, and thus the Polish middle class became \textit{compradores}. The key feature of semi-peripheral nations, then, is that they are both exploiters and exploited. Semiperipheral economies sit at the nexus between world systems, or between the frontier's extraction and the core's development.

The weakness of Wallerstein's theory is its scope. To falsify or confirm it would require near complete knowledge of all human activity, which might motivate research well but will always be playing catch-up to the \textit{now}. It will be interesting to see to whether surviving communist states such as Cuba or the PRC can actually defy his predictions. Secondly, it is worth considering if the US has moved into the position of World Empire since the end of the Cold War.

\section*{Hopf (1998)}

Constructivism brings post-modern scepticism to IR Theory and seeks to interrogate the subconscious assumptions made by all parties in the discourse. Post-modernists seek to "denaturalize" phenomena which theorists often take for granted, and this reveals that much of what is considered "normal" is indeed artifice. Conventional Constructivists do not go as far as critical theorists, who have an interest in overturning the artificial normality of oppressive structures. Conventional constructivists are content to merely observe, analyse, and report how social processes shape the course of politics, lest the quest for human liberation merely perpetuate the cycle of domination that has stifled it so far.

Hopf offers a research agenda that constructivist theorists should adopt in the new millennium. First, he presents a survey of constructivist interpretations of famous concepts in IR theory and uses novel approaches in classic IR puzzles. Constructivists do not claim to have all the answers; rather, they prefer eclectic and and inter-disciplinary methods to explain phenomena otherwise inexplicable with mainstream orthodoxy. Anticipating objections, Hopf points out that "Constructivism is no shortcut," since it recognises the inertia of entrenched power (which can't be simply "thought away") and it requires vast amounts of information to empirically analyse fuzzy concepts like "norms" and "practices" (in opposition to say, rational game theorists relying on radically simplified models or offensive realists tallying up military expenditures). My personal criticism of conventional Constructivism is its unwillingness to sacrifice ontological openness for a workable causal theory, as the critical theorists do. Unsurprisingly, I am already sympathetic to the critical theorists, but I concede Hopf's point that oftentimes class-reductionism is a problem within the discourse.

\end{multicols}

\end{document}
